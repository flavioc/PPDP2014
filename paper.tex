% This is LLNCS.DEM the demonstration file of
% the LaTeX macro package from Springer-Verlag
% for Lecture Notes in Computer Science,
% version 2.4 for LaTeX2e as of 16. April 2010
%
\documentclass{acm_proc_article-sp}
%
\usepackage{makeidx}  % allows for indexgeneration
\usepackage{fancyvrb}
\usepackage{amsmath}
\usepackage{graphicx}
\usepackage{caption}
\usepackage{subcaption}
\usepackage{wrapfig}
\usepackage{nth}
\usepackage[backref,pageanchor=true,plainpages=false, pdfpagelabels, bookmarks,bookmarksnumbered]{hyperref}
\input{fp-macros}

\begin{document}
\newcommand{\cmu}{\ensuremath{^\dag}}
\newcommand{\fcup}{\ensuremath{^\ddag}}
  \makeatletter
    \let\@copyrightspace\relax
      \makeatother
\numberofauthors{3} %  in this sample file, there are a *total*

\title{Design and implementation of a multithreaded virtual machine for executing linear logic programs}

\author{
\alignauthor
   Flavio Cruz \\
\affaddr{Carnegie Mellon University, Pittsburgh, PA 15213, USA}\\
\email{fmfernan@cs.cmu.edu} \\
\alignauthor Ricardo Rocha \\
\affaddr{CRACS \& INESC TEC, Faculty of Sciences, University Of Porto} \\
\affaddr{Rua do Campo Alegre, 1021/1055, 4169-007 Porto, Portugal} \\
\email{ricroc@dcc.fc.up.pt} \\
\alignauthor Seth Copen Goldstein \\
\affaddr{Carnegie Mellon University, Pittsburgh, PA 15213, USA}\\
\email{seth@cs.cmu.edu} \\
}

%\institute{Carnegie Mellon University, Pittsburgh, PA 15213, USA\\
%\email{\{fmfernan, seth\}@cs.cmu.edu}
%\and
%CRACS \& INESC TEC, Faculty of Sciences, University Of Porto\\
%Rua do Campo Alegre, 1021/1055, 4169-007 Porto, Portugal\\
%\email{\{flavioc, ricroc\}@dcc.fc.up.pt}}

\maketitle              % typeset the title of the contribution

\begin{abstract}
\begin{abstract}
Linear Meld is a concurrent forward-chaining linear logic programming
language where logical facts can be asserted and retracted in a
structured way. In Linear Meld, a program is seen as a database of
logical facts and a set of derivation rules. The database of facts is
partitioned by the nodes of a graph structure which leads to
parallelism when nodes are executed simultaneously. Due to the
foundations on linear logic, rules can retract facts in a declarative
and structured fashion, leading to more expressive programs. We
present the design and implementation of the virtual machine that we
implemented to run Linear Meld on multicores, with particular focus on
thread management, code organization, fact indexing, rule execution,
and database organization for efficient fact insertion, lookup and
deletion. Our results show that the virtual machine is
capable of scaling programs with up to 16 threads and also exhibits
interesting scalar performance results due to our indexing optimizations.
\end{abstract}

\category{D.1.3}{PROGRAMMING TECHNIQUES}{Concurrent Programming}[Parallel Programming]
\category{D.3.4}{PROCESSORS}{Interpreters}
\category{D.3.4}{PROCESSORS}{Run-time environments}

\terms{Design, Languages, Performance}

\keywords{Linear Logic, Virtual Machine, Implementation}

\keywords{linear logic programming, virtual machine, implementation}
\end{abstract}

\section{Introduction}
Logic programming is a declarative programming paradigm that has been used to advance the state of parallel programming.
Since logic programs are declarative, they are much easier to parallelize than imperative programs.
First, logic programs are easier to reason about since they are based on logical foundations.
Second, logic programmers do not need to use low level programming constructs such as locks or semaphores to
coordinate parallel execution, because logic systems hide such details from the programmer.

Logic programming languages split into two main fields: \emph{forward-chaining} and \emph{backwards-chaining} 
programming languages. Backwards-chaining logic programs are composed of a set of rules that can be activated by inputing a query. Given a query $q(\hat{x})$, an interpreter will work backwards by matching $q(\hat{x})$ against the head of a rule. If found, the interpreter will then try to match the body of the rule, recursively, until it finds the program axioms (rules without body). If the search procedure succeeds, the interpreter finds a valid substitution for the $\hat{x}$ variables. A popular backwards-chaining programming
language is Prolog~\cite{Colmerauer:1993:BP:154766.155362}, which has been a productive research language for executing logic
programs in parallel. Researchers took advantage of Prolog's non-determinism to evaluate subgoals
in parallel with models such as \emph{or-parallelism} and \emph{and-parallelism}~\cite{Gupta:2001:PEP:504083.504085}.

In a forward-chaining logic programming language, we start with a database of facts (filled with the program's
axioms) and a set of logical rules. Then, we use the facts of the database to fire the program's rules and derive new facts that are
then added to the database. This process is repeated many times until the database reaches \emph{quiescence} and no more information can
be derived from the program.
A popular forward-chaining programming language is Datalog~\cite{Ramakrishnan93asurvey}.

In this paper, we present a new forward-chaining logic programming language called Linear Meld (LM) that is specially suited
for concurrent programming over graphs. LM differs from Datalog-like languages because it integrates both classical
logic and linear logic into the language, allowing some facts to be retracted and asserted in a logical fashion. Although most
Datalog and Prolog-like programming languages allow some kind of state manipulation~\cite{Liu98extendingdatalog}, those features
are extra-logical and do not provide a structured way to manage state.

The roots of LM are the P2 system~\cite{Loo-condie-garofalakis-p2} and the original Meld~\cite{ashley-rollman-derosa-iros07wksp,ashley-rollman-iclp09}.
P2 was a Datalog-like language that mapped a computer network
to a graph, where each computer node could perform computations locally and communicate with neighbors.
Meld was itself inspired in the P2 system but adapted to the concept of massively distributed systems
made of modular robots with a dynamic topology.

LM also follows the same graph model of computation, but, instead, applies it to parallelize graph-based problems such as
graph algorithms, search algorithms and machine learning algorithms. LM programs are naturally concurrent since the graph of nodes
can be partitioned to be executed by different workers.
To realize LM, we have implemented a compiler and a virtual machine that executes LM programs on multicore machines\footnote{Source code is available at \url{http://github.com/flavioc/meld}.}.
We have implemented several parallel algorithms, including: belief propagation~\cite{Gonzalez+al:aistats09paraml},
belief propagation with residual splash~\cite{Gonzalez+al:aistats09paraml}, PageRank, graph coloring,
N-Queens, shortest path, diameter estimation, map reduce, quick-sort, neural network training, among others.

\iffalse
There are also non logical based systems intended to solve graph-based problems such as Dryad, Pregel or GraphLab.
The Dryad system~\cite{Isard:2007:DDD:1272996.1273005} is a framework that combines computational vertices
with communication channels (edges) to form a data-flow graph. 
The Pregel system~\cite{Malewicz:2010:PSL:1807167.1807184} is also graph-based, although programs must be represented
as a sequence of iterations where each iteration is composed of computation and message passing.
Finally, GraphLab~\cite{GraphLab2010} is a library for developing parallel (graph-based) machine learning algorithms in C++
and provides several schedulers to dictate the order of node execution.
%While Pregel uses message passing, GraphLab allows nodes to have read/write access to different scopes through different concurrent access models in order to balance performance and data consistency. Each consistency model provides different guarantees that are suited to multiple classes of algorithms.
\fi

As a forward-chaining linear logic programming language, LM shares similarities with Constraint Handling Rules (CHR)~\cite{Betz:2005kx,DBLP:journals/corr/abs-1006-3039}.
CHR is a concurrent committed-choice constraint language used to write constraint solvers. A CHR program is a set of rules and
a set of constraints (which can be seen as facts). Constraints can be consumed or generated during the application of rules.
Some optimizations ideas used in LM such as join optimizations and using different data structures for indexing facts
are inspired in research done in CHR~\cite{DBLP:journals/corr/cs-PL-0408025}.



This paper describes the current implementation of
our virtual machine and is organized as follows. First, we
briefly introduce the LM language. Then, we present an overview of the virtual machine, including code organization,
thread management, rule execution and database organization. Finally, we present preliminary results and
outline some conclusions.

\section{The LM Language}\label{lm_language}
\newcommand{\selector}[0]{[~S~\Rightarrow~y;~BE~] \lolli HE}
\newcommand{\comprehension}[0]{\{~\widehat{x};~BE;~SH~\}}
\newcommand{\aggregate}[0]{[~A~\Rightarrow~y;~\widehat{x};~BE;~SH_1;~SH_2~]}

%%%%%%%%%%%%%%%%%%%%%%%%%%%%%%%%%%%%%%%%%%%%%%%%%%%%%%%%%%%%%%%%%%%%%%

\section{The LM Language}
\label{lm_language}

Linear Meld (LM) is a logic programming language that offers a
declarative and structured way to manage mutable state.  A program consists of
a database of facts and a set of derivation rules. The database
includes persistent and linear facts. Persistent facts cannot be
deleted, while linear facts can be asserted and retracted.

The dynamic (or operational) semantics of LM is identical to
Datalog. Initially, we populate the database with the \emph{program's axioms} (initial facts)
and then determine which derivation rules can be applied by
using the current database. Once a rule is applied, new facts can be
derive, which are then added to the database. If a rule uses linear
facts, they are retracted from the database. The program stops when
\emph{quiescence} is achieved, i.e., when rules no longer apply.

Each fact is a predicate on a tuple of \emph{values}, where the type
of the predicate prescribes the types of the arguments.  LM rules are
type-checked using the predicate declarations in the header of the
program. LM has a simple type system that includes types such as
\emph{node}, \emph{int}, \emph{float}, \emph{string},
\emph{bool}. Recursive types such as \emph{list X} and \emph{pair X;
  Y} are also allowed.  Each rule in LM has a defined priority that is
inferred from its position in the source file.  Rules at the beginning
of the file have higher priority.
We consider all the new facts that have been not used yet to create a
set of \emph{candidate rules}.
The set of candidate rules is then applied (by priority)
and updated as new facts are derived.

%%%%%%%%%%%%%%%%%%%%%%%%%%%%%%%%%%%%%%%%%%%%%%%%%%%%%%%%%%%%%%%%%%%%%%

\subsection{Example}

\begin{figure}[t]
{\footnotesize
\begin{Verbatim}[numbers=right]
type left(node, node).
type right(node, node).
type linear value(node, int, string).
type linear replace(node, int, string).

// set of rules
replace(A, K, New),
value(A, K, Old)
   -o value(A, K, New). // we found our key

replace(A, RKey, RValue),
value(A, Key, Value),
!left(A, B),
RKey < Key
   -o value(A, Key, Value),
      replace(B, RKey, RValue). // go left

replace(A, RKey, RValue),
value(A, Key, Value),
!right(A, B),
RKey > Key
   -o value(A, Key, Value),
      replace(B, RKey, RValue). // go right

// initial configuration
!left(@0, @1).   !right(@0, @2).
!left(@1, @3).   !right(@1, @4). 
!left(@2, @5).   !right(@2, @6).

value(@0, 3, a).   value(@1, 1, b).
value(@2, 5, c).   value(@3, 0, d).
value(@4, 2, e).   value(@5, 4, f).
value(@6, 6, g).

// update key 6 to value x
replace(@0, 6, x).
\end{Verbatim}
}
\caption{LM program for replacing a key's value in a binary tree dictionary}
\label{code:btree_replace}
\end{figure}

We now present in Fig.~\ref{code:btree_replace}, a LM program that
implements the update operation for a binary tree dictionary
represented as key/value pairs. We first declare the predicates (lines
1-4) which represent the facts we are going to use. Predicate
\texttt{left/2} and \texttt{right/2} are persistent while predicates
\texttt{value/3} and \texttt{replace/3} are linear. Predicate
\texttt{value/3} assigns a key/value pair to a tree node and predicate
\texttt{replace/3} represents an update operation that updates the key
in the second argument to the value in the third argument.

The algorithm uses three rules for the three cases of updating a key's
value: the first rule performs the update (lines 6-9); the second rule
recursively picks the left branch for the update operation (lines
11-16); and the third rule picks the right branch (lines 18-23). The
initial axioms are presented in lines 26-33 and they describe the
initial binary tree configuration, including keys and values.
Finally, with the \texttt{replace(@0, 6, x)} axiom instantiated at the
root node $@0$, we intend to change the value of key 6 to value
`$x$'. Note that when writing rules or axioms, persistent facts are
preceded with a \texttt{!}.

\begin{figure*}[t]
\centering
\subfigure[Initial database]
   {\label{fig:btree_trace1}\includegraphics[width=0.44\textwidth]{figures/btree_trace1.pdf}}
\subfigure[After applying rule 3 at node $@0$]
   {\label{fig:btree_trace2}\includegraphics[width=0.44\textwidth]{figures/btree_trace2.pdf}}
\subfigure[After applying rule 3 at node $@2$]
   {\label{fig:btree_trace3}\includegraphics[width=0.44\textwidth]{figures/btree_trace3.pdf}}
\subfigure[After applying rule 1 at node $@6$]
   {\label{fig:btree_trace4}\includegraphics[width=0.44\textwidth]{figures/btree_trace4.pdf}}
\caption{An execution trace for the binary tree dictionary program}
\label{fig:btree_trace}
\end{figure*}

Figure~\ref{fig:btree_trace} illustrates the trace of the
execution. Note that the program database is partitioned by the tree
nodes using the first argument of each fact. In
Fig.~\ref{fig:btree_trace1}, we present the database filled with the
program's axioms. Next, we follow the right branch using rule 3 since
$6 > 3$ (Fig.~\ref{fig:btree_trace2}). We then use the same rule
again in Fig.~\ref{fig:btree_trace3} where we finally reach the key
6. Here, we apply rule 1 and \texttt{value(@6, 6, g)} is updated to
\texttt{value(@6, 6, x)} (Fig.~\ref{fig:btree_trace4}).

%%%%%%%%%%%%%%%%%%%%%%%%%%%%%%%%%%%%%%%%%%%%%%%%%%%%%%%%%%%%%%%%%%%%%%

\subsection{Syntax}

Table~\ref{tbl:ast} shows the abstract syntax for rules in LM.  An LM
program $Prog$ consists of a set of derivation rules $\Sigma$ and a
database $D$.  Each derivation rule $R$ can be written as $BE \lolli
HE$ where $BE$ is the body of a rule and $HE$ is the head. Rules
without bodies are allowed in LM and they are called
\textit{axioms}. Rules without heads are also allowed.  The body of a
rule, $BE$, may contain linear ($L$) and persistent ($P$) \emph{fact
  expressions} and constraints ($C$). Fact expressions are template
facts that instantiate variables (from facts in the
database). Variables can be used again in the body for matching and
also in the head when instantiating facts. Constraints are boolean
expressions that must be true in order for the rule to be
fired. Constraints use variables from fact expressions and are built
using a small functional language that includes mathematical
operations, boolean operations, external functions and literal values.
The head of a rule, $HE$, contains linear ($L$) and persistent ($P$)
\emph{fact templates} which are uninstantiated facts to derive new
facts. Head expressions may use the variables instantiated in the
body. The head can also have \emph{comprehensions} ($CE$) and
\emph{aggregates} ($AE$).

\begin{table}[ht]
\centering
{\scriptsize
\begin{tabular}{llcl}
Program          & $Prog$ & $::=$ & $\Sigma, D$ \\
Set of Rules     & $\Sigma$ & $::=$ & $\cdot \; | \; \Sigma, R$\\
Database         & $D$ & $::=$ & $\Gamma; \Delta$ \\
Rule             & $R$ & $::=$ & $BE \lolli HE \; | \; \forall_{x}. R$ \\
Body Expression  & $BE$ & $::=$ & $L \; | \; P \; | \; C \; | \; BE, BE \; | \; \exists_{x}. BE \; | \; 1$\\
Head Expression  & $HE$ & $::=$ & $L \; | \; P \; | \; HE, HE \; | \; CE \; | \; AE \; | \; 1$\\
Linear Fact      & $L$ & $::=$ & $l(\hat{x})$\\
Persistent Fact  & $P$ & $::=$ & $\bang p(\hat{x})$\\
Constraint       & $C$ & $::=$ & $c(\hat{x})$ \\
Comprehension    & $CE$ & $::=$ & $\comprehension$ \\
Aggregate        & $AE$ & $::=$ & $\aggregate$ \\
Operations       & $A$ & $::=$ & $\mathtt{min} \; | \; \mathtt{max} \; | \; \mathtt{sum} \; | \; \mathtt{count}$ \\
Sub-Head         & $SH$ & $::=$ & $L \; | \; P \; | \; SH, SH \; | \; 1$\\
Linear Facts     & $\Delta$ & $::=$ & $\cdot \; | \; \Delta, l(\hat{t})$ \\
Persistent Facts & $\Gamma$ & $::=$ & $\cdot \; | \; \Gamma, \bang p(\hat{t})$ \\
\end{tabular}
}
\caption{Abstract syntax of LM}
\label{tbl:ast}
\end{table}

We created the concept of comprehensions to be used when the
consumption of a linear fact should generate a set of facts
accordingly to the current contents of the database. In a
comprehension $\comprehension$, $\widehat{x}$ is a list of variables,
$BE$ is the body of the comprehension and $SH$ is the head. The body
$BE$ is used to generate all possible combinations for the head $SH$,
according to the facts in the database. The following program shows a
simple example that uses comprehensions:

{\footnotesize
\begin{Verbatim}
!edge(@1, @2).   !edge(@1, @3).
iterate(@1).

iterate(A)
   -o {B | !edge(A, B) | perform(B)}.
\end{Verbatim}
}

When the rule is fired, we consume \texttt{iterate(@1)} and then
generate the comprehension. Here, we iterate through all the
\texttt{edge/2} facts that match \texttt{!edge(@1, B)}, which are
\texttt{!edge(@1, @2)} and \texttt{!edge(@1, @3)}. For each fact, we
then derive \texttt{perform(B)}, namely \texttt{perform(@2)} and
\texttt{perform(@3)} in this example.

Another useful feature is the ability to reduce several facts into a
single fact. For that, we have aggregates, a special kind of sub-rule
that works very similarly to comprehensions. In a aggregate
$\aggregate$, $A$ is the aggregate operation, $\widehat{x}$ is the
list of variables introduced in $BE$, $SH_1$ and $SH_2$ and $y$ is the
variable in the body $BE$ that represents the values to be aggregated
using $A$. Like comprehensions, we use $\widehat{x}$ to try all the
combinations of $BE$, but, in addition to deriving $SH_1$ for each
combination, we aggregate the values represented by $y$ and derive
$SH_2$ only once using $y$. As an example, consider the following
program:

{\footnotesize
\begin{Verbatim}
price(@1, 3).   price(@1, 4).   price(@1, 5).
count-prices(@1).

count-prices(A)
   -o [sum => P | . | price(A, P) | 1 | total(A, P)].
\end{Verbatim}
}

By applying the rule, we consume \texttt{count-prices(@1)} and derive
the aggregate which consumes all the \texttt{price(@1, P)} linear
facts. These are summed up and \texttt{total(@1,~12)} is derived. LM
provides several aggregate operations, including the \emph{minimum},
\emph{maximum}, \emph{sum} and \emph{count}.

%%%%%%%%%%%%%%%%%%%%%%%%%%%%%%%%%%%%%%%%%%%%%%%%%%%%%%%%%%%%%%%%%%%%%%

\subsection{Concurrency}

LM is at its core a concurrent programming language. The database of
facts can be seen as a graph data structure where each node contains a
fraction of the database.  To accomplish this, we force the first
argument of each predicate to be typed as a \emph{node}. We then
restrict the derivation rules to only manipulate facts belonging to a
single node.  However, the expressions in the head may refer to other
nodes, as long as those nodes are instantiated in the body of the
rule.

Due to the restrictions on LM rules, nodes are able to run rules
independently without using other node's facts. Node computation
follows a \emph{don't care} or \emph{committed choice} non-determinism
since any node can be picked to run as long as it contains enough
facts to fire a derivation rule.  Facts coming from other nodes will
arrive in order of derivation but may be considered partially and
there is no particular order among the neighborhood. To improve
concurrency, the programmer is encouraged to write rules that take
advantage of the non-deterministic nature of execution.


\section{The Virtual Machine}\label{virtual_machine}

We developed a compiler that compiles LM programs to byte-code and a multi-threaded
virtual machine (VM) using the Pthreads library to run the byte-code.
The goal of our system is to keep the threads as busy as possible and to reduce inter-thread communication.

The load balancing aspect of the system is performed by our work scheduler that is based on a work
stealing algorithm. Initially, the system will partition the application
graph of $N$ nodes into $P$ subgraphs (the number of threads) and then each thread will work on their own subgraph.
During execution, threads can steal nodes of other threads to keep themselves busy.

Reduction of inter-thread communication is achieved by first ordering the node addresses present
in the code in such a way that closer nodes are clustered together and then partitioning them
to threads. During compilation, we take note of predicates that are used in rules for
communication rules (arguments with type \emph{node}) and then build a graph of nodes from
the program's axioms. The nodes of the graph are then ordered by using a breadth-first search
algorithm that changes the nodes of addresses to the domain $[0, n[$, where $n$ is the number of
nodes. Once the VM starts, we simply partition the range
$[0, n[$.

\subsection{Threads}

When the VM starts, it reads the byte-code file and starts all threads.
The partitioning of nodes is performed by dividing the domain of node addresses across the number of available threads.
In Fig.~\ref{fig:overview} we present the overview of a virtual machine running 2 threads and a program with 6 nodes.
Node that the dotted arrows represent the edges between nodes. In each thread space, we place the nodes owned by each thread.
Initially, the threads grab their own nodes and assign the \texttt{owner} property of each.
Because only one thread is allowed to do computation on a node at any giving time, the owner property
defines the thread with such permission.

\begin{figure*}[ht]
   \centering
   \scalebox{0.60}{\includegraphics[]{overview.pdf}}
   \caption{Overview of the virtual machine (showing 2 threads).}
   \label{fig:overview}
\end{figure*}

Included in the thread space is the \emph{Work Queue}, a single linked list containing \emph{active nodes}, i.e.,
nodes that have new facts to process. Initially, the \emph{Work Queue} is filled with all the nodes of the thread
in order to derive the axioms.

The main thread loop is shown in Fig.~\ref{code:work_loop}, where the threads
inspects their work queue for active nodes. Procedure \texttt{process\_node()} takes a node with new candidate rules
and executes them. If the work queue is empty, the thread attempts to steal one node from another thread before
becoming idle (\emph{Steal node} in Fig.~\ref{fig:overview}). Starting from a random thread, it cycles through all the threads to find one active node.
Eventually, there will be no more work to do and the threads will go idle. There is a global atomic counter, a global
boolean flag and one boolean flag for each thread that are used to detect termination.
Once a thread goes idle, it decrements the global counter and changes its flag to idle. If the counter
reaches zero, the global flag is set to idle. Since every thread will be busy-waiting and checking
the global flag, they will detect the change and stop executing.

\begin{figure}[h!]
\scriptsize\begin{Verbatim}
void work_loop(thread_id tid):
   while (true):
      current_node = NULL;
      if(has_work(tid)):
         current_node = pop_work(tid); // take node from the queue
      else:
         target_thread = random(NUM_THREADS);
         for (i = 0; i < NUM_THREADS && current_node == NULL; ++i):
            // need to steal a node
            target_thread = (target_thread + 1) % NUM_THREADS;
            current_node = steal_node_from_thread(target_thread)
      if(current_node == NULL):
         become_idle(tid);
         if(!synchronize_termination(tid)):
            return;
         become_active(tid);
      else:
         process_node(current_node, tid);
\end{Verbatim}
  \caption{Thread work loop.}
  \label{code:work_loop}
\end{figure}

\subsection{Nodes}

In Fig.~\ref{fig:overview}, we also present the internal structure of a node (node \texttt{@1}). Each node
contains the following: a database composed of \emph{Linear DB} (linear facts) and \emph{Persistent DB} (persistent facts) (further explained in \ref{sec:database});
rule matching structures (\emph{Rule Matching Structures}); and a \emph{Fact Buffer}, for storing intermediate facts coming
from other threads.

Whenever a new fact is derived through rule derivation, we need to update the data structures for the corresponding node.
This is trivial if the thread that owns the node derived the fact also. However, if that is not the case, then we have to synchronize
since multiple threads need to update the same data structures. We added a lock and a boolean flag to each node to protect the access to its data structures.
For example, in Fig.~\ref{fig:overview}, if thread 2 derives a fact to node \texttt{@1} (owned by thread 1), then thread 2 checks the node's flag
and if not activated, will lock node \texttt{@1} and then perform the required updates (\emph{New fact (1)}). If the flag is activated, it means
that thread 1 is currently executing the node, therefore it will not touch the main node data structures, but instead will add the new fact
to \emph{Fact Buffer} (\emph{New fact (2)}). The facts stored in \emph{Fact Buffer} will then be processed by thread 1 whenever the node is about to be executed.

Another thread interaction happens during fact derivation. In the example, after the fact is derived and sent to node \texttt{@1}, thread 2 also may need
to activate node \texttt{@1} by pushing it to the \emph{Work Queue} of thread 1 (\emph{Activate node}). Each node has another flag called \emph{In Queue} that indicates
if the node is currently placed in the \emph{Work Queue}. If the node is not in the queue, then we push it to the queue and then check if the target thread is currently idle.
After this synchronization point, the target thread will be active and with a new node to process.

\subsubsection{Database Data Structures}\label{sec:database}

We said before that LM rules are constrained by the first argument. Because nodes can execute
independently, our database is indexed by the node address and each sub-database does not
need to deal with synchronization issues since at any given point, only one thread will be using
the database. Note that the first argument of each fact is not stored.

The database must be implemented efficiently because during matching of rules we need
to restrict the facts using a given \emph{match object}, which fixes arguments of the target predicate to instantiated values.
Each sub-database is implemented using three kinds of data structures:

\begin{itemize}
   \item \emph{Trie Data Structures} are used exclusively to store persistent facts.
   Tries are trees where facts are indexed by the common arguments.
      
   \item \emph{Doubly Linked List Data Structures} are used to store linear facts.
   We use a double linked list because it is very efficient to add and remove facts.
   
   \item \emph{Hash Table Data Structures} are used to improve lookup when linked lists are too long and when we need to do search filtered by a fixed argument. The virtual machine decides which arguments are best to be indexed
   (see "Indexing") and then
   uses an hash table indexed by the appropriate argument. If we need to go through all the facts, we just iterate through all the facts in the table. For collisions, we use the above doubly linked list data structure.
\end{itemize}

\begin{figure}[]
   \centering
   \includegraphics[width=0.40\textwidth]{hash_table.pdf}
   \caption{\small{Hash table data structure for storing predicate \texttt{a(int,int)}.}}
   \label{fig:hash_table}
\end{figure}

Figure~\ref{fig:hash_table} shows an example for a hash table data structure
with 3 linear facts indexed by the second argument and stored as doubly linked list
in bucket \texttt{2}. Each linear fact contains the regular list pointers, a \texttt{flags} field
and the fact arguments. Those are all stored continuously to improve data
locality. One use of the \texttt{flags} field is to mark that a fact is already being used. For example,
consider the rule body \texttt{a(A,B), a(C,D) -o ...}. When we first pick a fact for \texttt{a(A, B)} from the hash table,
we mark it as being used in order to ensure that, when we retrieve facts for \texttt{a(C, D)}, the first one
cannot be used since that would violate linearity.

\subsubsection{Rule Engine}\label{rule_engine}

The rule engine decides which rules may need to be executed while taking into account rule priorities.
There are 5 main data structures for scheduling rule execution;
\texttt{Rule Queue} is the bitmap representing the rules that will be run;
\texttt{Active Bitmap} contains the rules that have enough facts to be fired;
\texttt{Inactive Bitmap} contains the rules that must be dropped from \texttt{Rule Queue};
\texttt{Predicates Bitmap} marks the newly derived facts;
and \texttt{Predicates Count} counts the number of facts per predicate.
To understand how our engine works, consider the following rules and axioms:

{\footnotesize\begin{Verbatim}
// rules
a, e(1) -o b.
a -o c.
b -o d.
e(0) -o f.
c -o e(1).

// axioms
a.
e(0).
\end{Verbatim}
}

\begin{figure*}[]
   \centering
   \scalebox{0.7}{\includegraphics[]{rule_queue1.pdf}}
   \caption{Program and rule engine data structures: before and after applying the \nth{2} rule.}
   \label{fig:rule_engine}
\end{figure*}


In our example, since we have facts \texttt{a} and \texttt{e(0)},
we will execute the second rule \texttt{a -o c}.
In order to pick such rule, we need to take the least significant bit from the \texttt{Rule Queue} bitmap.
Because the derivation is successful, we will consume \texttt{a} and derive \texttt{c}.
We thus mark the \texttt{c} predicate in the \texttt{Predicates Bitmap} and the first and second rules
in \texttt{Dropped Bitmap} since such rules are no longer applicable (\texttt{a} is gone). To update the \texttt{Rule Queue},
we remove the bits marked in \texttt{Dropped Bitmap} and add the active rules marked in \texttt{Active Bitmap} that are affected
by predicates in \texttt{Predicates Map}. The engine thus schedules the fourth and fifth rules to run (Fig.~\ref{fig:rule_engine}b).

Note that every node in the program has the same set of data structures present in Fig.~\ref{fig:rule_engine}.
We use 32 bits integers to implement bitmaps and an array of 16 bits integers to count facts, resulting in
$32 + 2P$ bytes per node, where $P$ is the number of predicates.

We do a small optimization to reduce the number of derivations of persistent facts. We
divide the program rules into two sets: \emph{persistent rules} and \emph{non persistent rules}.
Persistent rules are rules where only persistent facts are involved. We compile such rules
incrementally, that is, we attempt to fire all rules where a persistent fact is used. This is called
the \emph{pipelined semi-naive} evaluation and it originated in the P2 system~\cite{Loo-condie-garofalakis-p2}.
This evaluation method avoids excessing re-derivations of the same fact. The order of derivation does not matter for those rules, since
only persistent facts are used.

\subsection{Byte-Code}

A byte-code file contains meta-data about the program's predicates, initial nodes, partitioning
information, and code for each rule.
Each VM thread has 32 registers that are used during rule execution.
Registers can store facts, integers, floats, node addresses and pointers to runtime 
data structures (lists and structures). When registers store facts, we can reference
fields in the fact through the register.

Consider a rule \texttt{!a(X,Y), b(X,Z), c(X,Y) -o d(Y)} and a database with
\texttt{!a(1,2)}, \texttt{!a(2,3)}, \texttt{b(1,3)}, \texttt{b(5,3)}, \texttt{c(1,2)}, \texttt{c(1,3)},
\texttt{c(5,3)}. Rule execution proceeds in a series of recursive loops, as follows: the first loop retrieves an
iterator for the persistent facts of \texttt{!a/2} and moves the first valid fact, \texttt{!a(1,2)},
to register 0; the inner loop retrieves linear facts that match \texttt{b(1,Z)} (from the
\emph{join constraint}) and moves \texttt{b(1,3)} to register 1; in the final
loop we move \texttt{c(1,2)} to register 2 and the body of the rule is successfully matched. Next, we
derive \texttt{d(2)}, where \texttt{2} comes from register 0.
Fig.~\ref{fig:byte_code} shows the byte-code for this example.

\begin{figure}[]
\scriptsize\begin{Verbatim}
PERSISTENT ITERATE a MATCHING TO reg 0
  LINEAR ITERATE b MATCHING TO reg 1
      (match).0=0.0
    LINEAR ITERATE c MATCHING TO reg 2
        (match).0=0.0
        (match).1=0.1
      ALLOC d TO reg 3
      MVFIELDFIELD 0.1 TO 3.0
      ADDLINEAR reg 3
      REMOVE reg 2
      REMOVE reg 1
      TRY NEXT
    NEXT
  NEXT
RETURN
\end{Verbatim}
\caption{\small{Byte-code for rule \texttt{!a(X,Y), b(X,Z), c(X,Y) -o d(Y).}}}
\label{fig:byte_code}
\end{figure}

In case of failure, we jump to the previous outer loop in order to try the next candidate fact.
If a rule matches and the head is derived, we backtrack to the inner most \emph{valid loop}, i.e.,
the first inner loop that uses linear facts or, if there are no linear facts involved, to the previous 
inner loop. We need to jump to a valid loop because we may have loops with linear facts that are now invalid.
In our example, we would jump to the
loop of \texttt{b(X,Z)} and not \texttt{c(X,Y)}, since \texttt{b(1,3)} was consumed.

\begin{figure}[]
\scriptsize\begin{Verbatim}
LINEAR ITERATE a MATCHING TO reg 0
  MVFIELDREG 0.0 TO reg 1
  MVINTREG INT 1 TO reg 2
  reg 1 INT PLUS reg 2 TO reg 3
  MVREGFIELD reg 3 TO 0.0
  UPDATE reg 0
  TRY NEXT
RETURN
\end{Verbatim}
\caption{\small{Byte-code for rule \texttt{a(N) -o a(N+1)}.}}
\label{code:update}
\end{figure}

The compiler re-orders the fact expressions used in the body in order to make execution more
efficient. For example, it forces the join constraints in rules to appear at the beginning so
that matching will fail sooner rather than later. It also does the same for constraints.
Note that for every loop, the compiler adds the \emph{match object}, which contains information
about which arguments need to match, so that runtime matching is efficient.

Our compiler also detects cases where we re-derive a linear fact with new arguments.
For example, as shown in Fig.~\ref{code:update}, the rule \texttt{a(N) -o a(N+1)}
will compile to code that reuses the old \texttt{a(N)} fact.
We use a \texttt{flags} field to mark updated nodes (presented next).

\subsection{Indexing}\label{indexing}

To improve fact lookup,
the VM employs a fully dynamic mechanism to decide which argument may be optimal to index.
The algorithm is performed in the beginning of the execution and empirically tries to assess the argument
of each predicate that more equally spreads the database across the values of the argument.
A single thread performs the algorithm for all predicates.

The indexing algorithm is performed in three main steps. First, it gathers statistics of lookup data by keeping a counter
for each predicate's argument.
Every time a fact search is performed where arguments are fixed to a value, the counter of such arguments is incremented. This phase is performed during rule execution for a small fraction of the nodes in the program.

The second step of the algorithm then decides the candidate arguments of each predicate.
If a predicate was not searched with any fixed arguments, then it will be not indexed.
If only one argument was fixed, then such argument is set as the indexing argument. Otherwise, the top 2 arguments
are selected for the third phase, where \emph{entropy statistics} are collected dynamically.

During the third phase, each candidate argument has an entropy score.
Before a node is executed, the facts of the target predicate
are used in the following formula applied for the two arguments:

{\scriptsize
\[
Entropy(A, F) = - \sum_{v \in values(F, A)} \frac{count(F, A = v)}{total(F)} 	\log_2 \frac{count(F, A = v)}{total(F)}
\]
}

Where $A$ is the target argument, $F$ is the multi-set of linear facts for the target predicate, $values(F, A)$ is set of values of the argument $A$, $count(F, A = v)$ counts the number
of linear facts where argument $A$ is equal to $v$ and $total(F)$ counts the number of linear facts in $F$.
The entropy value is a good metric because it tells us how much information is needed to describe an argument.
If more information is needed, then that must be the best argument to index.

For one of the arguments to score, $Entropy(A, F)$ multiplied by the number of times it has been used for lookup must be larger than the other argument.

The argument with the best score is selected and then
a global variable called \texttt{indexing\_epoch} is updated.
In order to convert the node's linked lists into hash tables, each node also has a local variable called \texttt{indexing\_epoch}
that is compared to the global variable in order to rebuild the node database according to the new indexing
information.

Our VM also dynamically resizes the hash table if necessary. When the hash table becomes
too dense, it is resized to the double. When it becomes too sparse, it is reduced in half
or simply transformed back into a doubly linked list. This is done once in a while, before a node executes.

We have seen very good results with this scheme. For example, for the all-pairs shortest paths program,
we obtained a 2 to 5-fold improvement in sequential execution time.
The overhead of dynamic indexing is negligible since programs run almost as fast
as if the indices have been added from the start.

\subsection{Runtime Data Structures}

LM supports recursive types such as lists and pairs. These complex data structures are stored in
the heap of the VM and are managed through reference counting. For instance, each list
is a \emph{cons cell} with 3 fields: \texttt{tail}, the pointer to the next element of the list;
\texttt{head}, the element stored by this element of the list; and \texttt{refs} that counts the
number of pointers to this list element in the VM. The list is deleted from the heap whenever
\texttt{refs} is decremented to 0.


\section{Experimental Results}\label{results}
This section presents preliminary results for our VM.
First, we present scalability results in order to show that LM programs can take advantage of multicore architectures.
Next, we present a comparison with similar programs written in other programming languages
in order to show evidence that our VM is viable.

For our experimental setup, we used a machine 
with two 16 (32) Core AMD Opteron
(tm) Processor 6274 $@$ 2.2 GHz with 32 GBytes of RAM memory and running the Linux
kernel 3.8.3-1.fc17.x86\_64.
     We compiled our VM using GCC 4.7.2 (g++) with the flags \texttt{-O3 -std=c+0x -march=x86-64}.
     We run all experiments 3 times and then averaged the execution time.
     
\subsection{Scalability Results}

For this section, we run each program using 1, 2, 4, 6, 8, 10, 12, 14 and 16 threads and compared the runtime against the execution of the sequential version of the VM. We used the following programs:

\newcommand{\figsize}[0]{6.5cm}
\captionsetup[sub]{              % The subcaption settings
       font=scriptsize}

\begin{itemize}
   \item Greedy Graph Coloring (GGC) colors nodes in a graph so that no two adjacent nodes have the same color. We start with a small number of colors and then we expand the number of colors when we cannot color the graph.
   \item PageRank implements a PageRank algorithm without synchronization between iterations. Every time a node sends a new rank to its neighbors and the change was significant, the neighbors are scheduled to recompute their ranks.
   \item N-Queens, the classic puzzle for a 13x13 board.
   \item Belief Propagation, a machine learning algorithm to denoise images.
\end{itemize}

Figure~\ref{exp:graph_coloring} presents the speedup results for the GGC program using 2 different datasets. In Fig.~\ref{exp:graph_coloring}a we show the speedup for a search engine graph of 12,000 webpages\footnote{Available from \url{http://www.cs.toronto.edu/~tsap/experiments/download/download.html}}. Since this dataset follows the power law, that is, there is a small number of pages with a lots of links (1\% of the nodes have 75\% of the edges), the speedup is slightly worse than the benchmark shown in Fig.~\ref{exp:graph_coloring}b, where we use a random dataset of 2,000 nodes with an uniform distribution of edges.

\newcommand{\plotsize}{0.38\textwidth}

\begin{figure*}[t]
   \centering
   \begin{subfigure}[b]{\plotsize}
      \includegraphics[width=\textwidth]{speedup_greedy-graph-coloring-search_engines.pdf}
      \caption{Using a graph of web pages with around 12,000 nodes and 292,000 edges.}
   \end{subfigure}
   ~~
   \begin{subfigure}[b]{\plotsize}
      \includegraphics[width=\textwidth]{speedup_greedy-graph-coloring-2000.pdf}
      \caption{Using a random graph with 2,000 nodes and 600,000 edges.}
   \end{subfigure}
   \caption{Experimental results for the GGC algorithm.}
   \label{exp:graph_coloring}
\end{figure*}

The PageRank results are shown in Fig.~\ref{exp:pagerank}. We used the same search engine dataset as before and a new random dataset with 5,000 nodes and 500,000 edges. Although the search engine graph (Fig.~\ref{exp:pagerank}a) has half the edges (around 250,000), it scales better than the random graph (Fig.~\ref{exp:pagerank}b), meaning that the PageRank program depends on the number of nodes to be more scalable.

\begin{figure*}[ht]
   \centering
   \begin{subfigure}[b]{\plotsize}
      \includegraphics[width=\textwidth]{speedup_pagerank-search_engines.pdf}
      \caption{Using a graph of web pages collected from a search engine (around 12,000 nodes and 292,000 edges)}
   \end{subfigure}
   ~~
   \begin{subfigure}[b]{\plotsize}
      \includegraphics[width=\textwidth]{speedup_pagerank-5000.pdf}
      \caption{Using a random graph with 5,000 nodes and 500,000 edges.}
   \end{subfigure}
   \caption{Experimental results for the asynchronous PageRank algorithm.}
   \label{exp:pagerank}
\end{figure*}

The results for the N-Queens program are shown in Fig.~\ref{exp:8queens}. The program is not regular since computation starts at the top of the grid and then rolls down, until only the last row be doing computation. Because the number of valid states for the nodes in the upper rows is much less than the nodes in the lower rows, this may potentially lead to
load balancing problems. The results show that our system is able to scale well due to work stealing.

\begin{figure*}[ht]
   \centering
   \begin{subfigure}[b]{\plotsize}
      \includegraphics[width=\textwidth]{speedup_8queens-13.pdf}
      \caption{N-Queens program (13x13~board).}
      \label{exp:8queens}
   \end{subfigure}
   ~~
   \begin{subfigure}[b]{\plotsize}
      \includegraphics[width=\textwidth]{speedup_bp-400.pdf}
      \caption{BP (400x400~image).}
      \label{exp:bp}
   \end{subfigure}
   \caption{Experimental Results for N-Queens and Belief Propagation.}
\end{figure*}

Finally, we shown the results for the Belief Propagation~(BP) program in Fig.~\ref{exp:bp}. BP is a regular and asynchronous program
and benefits (as expected) from having multiple threads executing since the belief values of each node will converge faster.
The super-linear results prove this assertion.

\subsection{Absolute Execution Time}

As we have seen, our VM scales reasonably well, but how does it compare in terms of absolute execution time
with other competing systems? We next present such comparison for the execution time using one thread.

In Fig.~\ref{comp:nqueens} we compare the LM's N-Queens version against 3 other versions: a straightforward sequential program implemented
in C using backtracking; a sequential Python~\cite{vanRossum:1995:PRM}
implementation; and a Prolog implementation executed in
YAP Prolog~\cite{DBLP:journals/corr/abs-1102-3896}, an efficient implementation of Prolog. Numbers less than 1 mean that LM
is faster and larger than 1 mean that LM is slower. We see that LM easily beats Python, but is 5 to 10 times slower than YAP
and around 15 times slower than C.
However, note that if we use at least 16 threads in LM, we can beat the sequential implementation written in C.

\begin{figure*}[ht]
   \centering
   
   \begin{subfigure}[b]{0.45\textwidth}
      \resizebox{7cm}{!} {
      \begin{tabular}{ | l | l | l | l |}
       \hline

       Size & C & Python & YAP Prolog \\ \hline\hline
       10 & 16.92 & \textbf{0,62} & 5,42 \\
       11 & 21.59 & \textbf{0.64} & 6.47 \\
       12 & 10.32 & \textbf{0.73} & 7.61 \\
       13 & 14.35 & \textbf{0.88} & 10.38 \\
       \hline
       \end{tabular}}
       \caption{N-Queens problem.}
      \label{comp:nqueens}
   \end{subfigure}
   ~~
   \begin{subfigure}[b]{0.45\textwidth}
      \resizebox{6.5cm}{!} {
      \begin{tabular}{ | l | l | l | l |}
       \hline

       Size & C & Python & GraphLab \\ \hline\hline
       10 & \textbf{0.67} & \textbf{0,03} & \textbf{1.00} \\
       50 & \textit{1.77} & \textbf{0.04} & \textit{1.73} \\
       200 & \textit{1.99} & \textbf{0.05} & \textit{1.79} \\
       400 & \textit{2.00} & \textbf{0.04} & \textit{1.80} \\
       \hline
       \end{tabular}}
       \caption{Belief Propagation program.}
       \label{comp:bp}
   \end{subfigure}
   \caption{Comparing absolute execution times (System Time / LM Time).}
\end{figure*}

In Fig.~\ref{comp:bp} we compare LM's Belief Propagation program against a sequential C version, a Python version and a GraphLab version. GraphLab~\cite{GraphLab2010} is a parallel C++ library used to solve graph-based problems in machine learning. C and GraphLab
perform about the same since both use C/C++. Python runs very slowly since it is a dynamic programming language and BP has many
mathematical computations. We should note, however, that the LM version uses some external functions written in C++ in order
to improve execution time, therefore the comparison is not totally fair.

We also compared the PageRank program against a similar GraphLab version and LM is around 4 to 6 times slower.
Finally, our version of the all-pairs shortest distance algorithm is 50 times slower than a C sequential implementation of the Dijkstra algorithm, but it is almost twice as fast when compared to the same implementation in Python.


\section{Related Work}\label{related_work}
Virtual machines are a popular technique for implementing interpreters for high level programming languages.
Due to the increased availability of parallel machines and distributed architectures, several machine
models have been developed with parallelism in mind~\cite{Kara:1997:AMM:265274}.
One example of such machine is the Parallel Virtual Machine (PVM)~\cite{Sunderam90pvm:a}, which serves as an abstraction
to program heterogeneous computers as a single machine. Another important machine is the Threaded Abstract Machine (TAM)~\cite{CullerGSvE93,goldstein-tr94},
which defines a self-scheduled machine language of parallel threads where a program is represented as conventional control flow.

Prolog, the most prominent logic programming language, has a rich history of virtual machine research centered
around the Warren Abstract Machine WAM~\cite{AICPub641:1983}. The WAM offers special purpose instructions, including
unification instructions for different kinds of data and control flow instructions to implement backtracking.
The WAM is fully sequential and uses four memory areas: heap, stack, trail and the push down list.
Much research has been done to improve the speed and efficiency of the original WAM design~\cite{Costa07demand-drivenindexing,167005,Turk-Logic:1986fk}.

Prolog is naturally parallel because several clauses for the same goal (AND-parallelism) or all goals in a clause (OR-parallelism) can be tried
in parallel. Different abstract machines for AND-parallelism has been developed on top of the WAM~\cite{Hermenegildo:1986:AMB:913061,Lin:1988:AEL:900478}.
For OR-parallelism we have several models such as: the SRI model~\cite{Warren:1987:OEM:67683.67699},
the Argonne model~\cite{ButlerDLOOS88}, the MUSE model~\cite{Ali:1990fk} and the BC machine~\cite{Ali88}. While all those models have been
developed using the WAM, there are some parallel machines totally different from the WAM such as the PPAM~\cite{Kacsuk:1990:EMP:533578},
which is based on a data-flow model.


\section{Conclusions}
We have presented a parallel virtual machine for executing forward-chaining linear logic programs, with particular focus
on parallel scheduling on multicores, rule execution and database organization for fast insertion,
lookup, and deletion of linear facts.
Our experimental results show that the VM is able to scale the execution of programs when run with up to 16 threads.
Although this is still a work in progress, the VM also fairs well against other programming languages.
Moreover, since LM programs are concurrent by default, we can easily get better performance from the start by executing
them with multiple threads.

In the future, we intend to take advantage of linear logic to perform whole-program optimizations,
including computing program invariants, loop detection in rules and bypass of rule priorities. For instance,
we want to detect during compile time which might be the best data structure for each predicate depending in how it
is used in the rules.


\bibliographystyle{abbrv}
\bibliography{refs}
\end{document}

