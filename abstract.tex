\begin{abstract}
Linear Meld is a concurrent forward-chaining linear logic programming
language where logical facts can be asserted and retracted in a
structured way. In Linear Meld, a program is seen as a database of
logical facts and a set of derivation rules. The database of facts is
partitioned by the nodes of a graph structure which leads to
parallelism when nodes are executed simultaneously. Due to the
foundations on linear logic, rules can retract facts in a declarative
and structured fashion, leading to more expressive programs. We
present the design and implementation of the virtual machine that we
implemented to run Linear Meld on multicores, with particular focus on
thread management, code organization, fact indexing, rule execution,
and database organization for efficient fact insertion, lookup and
deletion. Our results show that the virtual machine is
capable of scaling programs with up to 16 threads and also exhibits
interesting scalar performance results due to our indexing optimizations.
\end{abstract}

\category{D.3.4}{PROCESSORS}{Interpreters}
\category{D.3.4}{PROCESSORS}{Run-time environments}
\category{D.1.3}{PROGRAMMING TECHNIQUES}{Concurrent Programming}[Parallel Programming]

\terms{Design, Languages, Performance}

\keywords{Linear Logic, Virtual Machine, Implementation}
