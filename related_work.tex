Virtual machines are a popular technique for implementing interpreters for high level programming languages.
Due to the increased availability of parallel machines and distributed architectures, several machine
models have been developed with parallelism in mind~\cite{Kara:1997:AMM:265274}.
One example of such machine is the Parallel Virtual Machine (PVM)~\cite{Sunderam90pvm:a}, which serves as an abstraction
to program heterogeneous computers as a single machine. Another important machine is the Threaded Abstract Machine (TAM)~\cite{CullerGSvE93,goldstein-tr94},
which defines a self-scheduled machine language of parallel threads where a program is represented as conventional control flow.

Prolog, the most prominent logic programming language, has a rich history of virtual machine research centered
around the Warren Abstract Machine WAM~\cite{AICPub641:1983}. The WAM offers special purpose instructions, including
unification instructions for different kinds of data and control flow instructions to implement backtracking.
The WAM is fully sequential and uses four memory areas: heap, stack, trail and the push down list.
Much research has been done to improve the speed and efficiency of the original WAM design~\cite{Costa07demand-drivenindexing,167005,Turk-Logic:1986fk}.

Prolog is naturally parallel because several clauses for the same goal (AND-parallelism) or all goals in a clause (OR-parallelism) can be tried
in parallel. Different abstract machines for AND-parallelism has been developed on top of the WAM~\cite{Hermenegildo:1986:AMB:913061,Lin:1988:AEL:900478}.
For OR-parallelism we have several models such as: the SRI model~\cite{Warren:1987:OEM:67683.67699},
the Argonne model~\cite{ButlerDLOOS88}, the MUSE model~\cite{Ali:1990fk} and the BC machine~\cite{Ali88}. While all those models have been
developed using the WAM, there are some parallel machines totally different from the WAM such as the PPAM~\cite{Kacsuk:1990:EMP:533578},
which is based on a data-flow model.
